\chapter{Kale commands}
\newcommand{\shortcut}[3]{\section[#1]{#1 \hfill #2}\label{cmd:#3}}
\newcommand{\pskip}[1]{{\bigskip\par\noindent #1}}

Commands different by only a shift key \keys{\shift} are closely
related. Commands that change the selection are usually not listed in the
context menu. Where applicable mnemonics are represented by a \textbf{B}old
letter.

\shortcut{Up / Down}{\ak{^} / \ak{v}}{up_down}
Change the selection to the previous or next expression by pre-order traversal.
This means that by running a command repeatedly, you will move through every
possible selection to either side of the cursor. If there is no previous or
next expression, the current selection is preserved.
\pskip{Alternative shortcuts: \keys{K}~/~\keys{J} (based on the Vim editor).}

\shortcut{Left / Right}{\ak{<} / \ak{>}}{left_right}
Change the selection to previous or next non-inline non-list expression by
pre-order traversal. This approximates moving between expressions that visually
resemble lines, or broken up lines. If there is no previous or next expression,
the current selection is preserved.
\pskip{Alternative shortcuts: \keys{H}~/~\keys{L} (based on the Vim editor).}

\shortcut{Add Space or Move Space Up}{\keys{\SPACE}}{smart_space}
This is also known as the \say{smart space}. If the currently selected
expression is a blank, \hyperref[cmd:move_up]{\say{Move Up}} is ran on the
current selection. If the selection is a function call, a new child space
is added in the first argument. Otherwise a new sibling space is inserted to
the right of the current selection. 

Because this command will not create a new sibling to function calls, the 
\hyperref[cmd:new_line]{\say{New Line Below /Above}}
\keys{N}~/~\keys{\shift + N} 
command is also useful to know.

\shortcut{Edit}{\keys{\return}}{edit}
This can also be invoked by double clicking on an expression. Edits the text
inside the expression.

If the expression is a space, instead opens a new menu,
letting you select the type of expression with which to replace the space. 
Once a selection is made editing proceeds as normal, with the exception of
editing function expressions, which when the edit is complete, will create
an appropriate number of spaces for their arguments.

If the expression is a list, no action is performed.

\pskip{Alternative shortcut: \keys{E}}

\shortcut{Copy}{\keys{C}}{copy}
Copy the currently selected expression to the top of the Clipboard Stack

\shortcut{Select Parent}{\keys{P}}{parent}
Select the parent of the currently selected expression. If a parent does not
exist, the current selection is preserved.

\shortcut{Move Up}{\keys{\shift + P}}{move_up}
Move the currently selected expression to be the last sibling of its parent.
If a parent does not exist, no action is performed.
This is similar to the "Barf" operation in the Emacs Par Edit mode.


\shortcut{New Line Below / Above}{\keys{N} / \keys{\shift + N}}{new_line}
Inserts a new list expression around the current selection. Note that "list
merging" is performed, merging immediate list children of a list with their
parent. This means that, for example, invoking this command on an expression
which already has a list as its parent, will not create a new list, instead
appending a new sibling to the selection to the current list.

\shortcut{Delete}{\keys{\backspace}}{delete}
Delete the currently selected expression, potentially replacing it with a new
space if no other expression would remain in a function.

\pskip{Alternative shortcut: \keys{D}}

\shortcut{Cut}{\keys{X}}{cut}
\hyperref[cmd:copy]{Copy} the currently selected expression, then
\hyperref[cmd:delete]{delete} it. Similar to the \keys{\ctrl + X} command in
text editors.

\shortcut{Delete and Add Space}{\keys{R}}{delete_blank}

Delete the currently selected expression, replacing it with a new space. This
helps helps if you want to completetly \textbf{R}eplace an expression with
something new.

\shortcut{Cut and Add Space}{\keys{S}}{cut_blank}
\hyperref[cmd:copy]{Copy} the currently selected expression, then perform
\hyperref[cmd:delete_blank]{\say{Delete and Add Space}}. This helps if you want
to \textbf{S}huffle expressions around, replacing the current selection with
something new, but using the old expression somewhere else.

\shortcut{Open Definition}{\keys{O}}{open_def}
Opens an editor with the function name of the currently selected expression.
If a function with a given name does not exist in the current workspace, one is
created.
If an editor with a given function is already opened, no new editor is created,
instead focus is moved to the closest editor displaying that function.

\pskip{This command can also be triggered by middle-clicking the expression.}

\shortcut{New Argument After / Before}{\keys{I} / \keys{\shift + I}}{new_arg}
Creates a new sibling space before or after the currently selected
expression. This is command largely superseded by the
\hyperref[cmd:smart_space]{\say{Smart Space}} \keys{\SPACE} action, but can be
useful for creating multiple consecutive spaces or if a sibling is desired
instead of a child and the currently selected expression is a function call.

\shortcut{Comment}{\keys{Q}}{comment}
Edits the comment on the currently selected expression. If no comment exists,
one is created. Note that comments on spaces and literals are handled
specially. An empty comment is automatically removed.

\shortcut{Disable}{\keys{\textbackslash}}{disable}
Disable the currently selected expression, enabling it if it is already
disabled. Spaces cannot be disabled.

\shortcut{Make a Variable...}{\keys{V}}{make_var}
Replaces the currently selected expression with a new variable expression, then
performs the \hyperref[cmd:edit]{\say{Edit}}~\keys{\return} command.

\shortcut{Make a String...}{\keys{G}}{make_string}
Replaces the currently selected expression with a new string literal
expression, then
performs the \hyperref[cmd:edit]{\say{Edit}}~\keys{\return} command.

\shortcut{Turn into a Function Call}{\keys{F}}{make_function}


\shortcut{Replace the Parent}{\keys{\shift + F}}{replace_parent}
%TODO: Document these