\documentclass[11pt]{report}

% Add url support for web pages etc.
\usepackage[square, numbers]{natbib}
% For custom spacing
\usepackage{setspace}
\setstretch{1.15}
% For the \say command
\usepackage{dirtytalk}
% Figures etc.
\usepackage{graphicx}
% Draft watermark.
\usepackage{draftwatermark}
% Acronym support.
\usepackage{acro}
% Extra appendix control; Add appendices to the toc and add a blank page.
\usepackage[page, toc]{appendix}
% Figure wrapping.
\usepackage{wrapfig}
% Additional control over itemize.
\usepackage{enumitem}
% PDFs links. Must be loaded before geometry, but as late as possible.
\usepackage{hyperref}
\hypersetup{
	colorlinks,
    allcolors=blue,
}
% As required by the handbook.
\usepackage[a4paper, margin=25mm]{geometry}
% Menus and menu keys. Recommended to load late.
\usepackage[os=win]{menukeys}

\title{Kale - An attempt at an accessible and powerful visual programming
environment}
\author{Maciej Goszczycki - 16316981}
\date{}

% Acronyms must be defined before the document starts.
\DeclareAcronym{svg}{short=SVG, long=Scalable Vector Graphics}
\DeclareAcronym{ui}{short=UI, long=user interface}
\DeclareAcronym{dom}{short=DOM, long=Document Object Model}
\DeclareAcronym{jsx}{short=JSX, long=JavaScript XML}

\begin{document}

\maketitle
% This must be defined before TOC, because the shortcut appendix uses it
% (even if it doesn't appear in the actual toc)
\newcommand{\ak}[1]{\keys{\arrowkey{#1}}}
\tableofcontents
\clearpage

% Do not clear page after every chapter
\begingroup
\let\clearpage\relax

\chapter{Abstract}
Contemporary text-based programming languages require the programer to be
constantly vigilant against syntax errors. This affects both novice and
experienced programmers. One obvious way of eliminating syntax errors is
get rid of text. While there exist many visual programming environments, many
eschew powerful editing capabilities for being beginner friendly.
This project explores how a Lisp-based visual programming environment might be
used to mitigate syntax issues while remaining attractive to both novices and
professional programmers.

\chapter{Introduction}

\section{Motivation}
Syntax errors are a fact of life in the programming industry. Programmers from
novice\cite{Denny2011} to professional spend significant time fighting or
avoiding syntax errors. It is clear a visual programming environment such
%TODO: Cite scratch, cite syntax errors among professionals
Scratch could eliminate syntax errors. There exists a large body of existing
visual programming environments \cite{Beldie1983}, but most focus exclusively
on making programming accessible to children and young adults. Most
professional tools focus instead of error-detection, ignoring visual means of
editing programs as slow and cumbersome. It should be possible to create a
visual programming environment friendly to novices, but powerful enough to be
taken seriously by professionals. 

This project implements a web based visual programming environment, designed
from the ground up to fit many skill levels. Kale should demonstrate that a
drag and drop/blocks style interface can co-exist alongside a keyboard driven
professional-focused editing experience. It should generate usable code,
runnable from within the interface.

\section{Approach}

%TODO: Write these out
\begin{itemize}
	\item Implementing the rendering engine
	\item Keyboard shortcuts
	\item Drag and drop functionality
	\item Working interpreter
\end{itemize}
Lisp's highly uniform syntax, makes it the perfect 

\section{Metrics}
\section{Project}

As part of the development process I contributed patches to two different
open-source projects, including adding \texttt{\#rrggbbaa} notation support
to the Popmotion library \cite{github-pr-popmotion} and updating
\say{styled-components} TypeScript typings to the a new major 5.0
version\cite{github-pr-styled-components}.

\chapter{Technical Background}

\section{Topic material}

\subsection{Visual Programming Environments}


\subsection{Command Composition}
Composing user commands is an important concern for any code editor. Chodarev
\cite{Chodarev2016} discussed this in length, pointing out that casual
text-based editing environments rely on simple commands that operate only on
elementary objects, manipulating single character at a time. However this
approach is quite inefficient, leading leading apps to adopt a secondary set of
ad-hoc commands for performing specific combinations of operations and
high-level objects (like deleting a word). One proposed alternative are
Vim-like shortcuts, where the user combines a smaller set of motion and action
commands to work on high-level objects.

\subsection{Proportional fonts}
Studies like Campbell et al. \cite{Campbell1981} and Beldie et al.
\cite{Beldie1983} have shown that proportional (variable-pitch) fonts are
faster to read than their \texttt{monospaced} counterparts. One notable exaple
of a proportional font being used typeset code is 

\section{Technical material}

\subsection{TypeScript}
TypeScript is a structurally and gradually typed superset of JavaScript being
developed at Microsoft \cite{Typescript}. It provides stronger type-safety
guarantees than JavaScript and allows programs to use modern and experimental
JavaScript features, such as class properties or optional chaining. 


\subsection{React and related libraries}
To ease the burden of manually manipulating the \ac{dom}, Kale uses the React
\cite{React} JavaScript library. React is a self-proclaimed \say{JavaScript
library for making user interfaces} being developed by Facebook. It
allows the program to write rendering
code in a functional fashion, relying a \say{Virtual \ac{dom} to make \ac{dom}
updates efficient}. Together with \ac{jsx}, a JavaScript extension supported by
TypeScript, it makes it easy to write complex \acp{ui} for the browser.

\subsection{\acf{svg}}
\ac{svg} is a web technology which can be used to display vector images in the
browser. It is scriptable by JavaScript and styleable with CSS, making it the
perfect fit for dynamically rendering any highly custom \acp{ui}. It is used by
Scratch for rendering their blocks interface.

\chapter{The Problem}

The programming language which Kale users will be writing should map well to
the UI, but be powerful enough to be able to express complex programs. It need
not necessarily be an existing language, it fact it might not be desirable to
make it one.

A proficient user should be able to write programs and navigate Kale without
having to reach for the mouse. This might involve mapping Kale concepts to
standard shortcut keys as well as developing new ones. It should be possible
for users of existing programming environments to adjust to Kale with relative
ease. Programs written in Kale should be readable, balancing accessibility and
information density.

Novice users should be able to to construct Kale programs in a simple and
initiative manner, on every device form-factor, whether touch-screen or
mouse-driven. It should be simple for them to discover new functionality and
edit existing programs without knowledge of higher level operations.

Effectively manipulating Lisp programs often requires commands for high-level
hierarchical selections and actions, which are often difficult for users to
remember. Editors like Vim use the command composition
pattern to enable these actions, but this comes at a great cost to the learning
difficulty \cite{Chodarev2016}

\chapter{The Solution}

\section{Expression structure}

\section{Drag and drop}

One of the main changes Kale introduces compared to a normal Lisp editor is
drag and drop. It lets novice users effectively manipulate Kale programs,
without prior knowledge of commands like copy and paste. Kale distinguished
between two types of drag and drop: Replacement and Insertion. 

\section{Layout}

One of the key ideas in Kale is that of structural underlines. As a
fundamental unit of nesting, parentheses present a number of challenges to a
user using drag and drop.
%TODO Talk about in inline, non-inline, underlines optional bubbles.

\section{Command structure}

Because Kale does not operate on elementary text elements, even the most
basic
commands can be much higher-level than ordinary editors, since instead of a
current cursor position, Kale operates on the selected expression. Even so,
figuring out the correct set of commands proved to be a challenge.

A good example of the difficulties that arise are the arrow key commands
\ak{^} \ak{v} \ak{<} \ak{>}. Initially these were implemented as
fundamental tree operations: \say{Select parent}, \say{Select first child},
\say{Select left sibling}, and \say{Select right sibling} respectively.
While logical, these operations were unintuitive to every user
Kale was shown too, including the author. In the end \say{smart selection} was
implemented, where \ak{<} and \ak{>} use the pre-order

\footnote{The preorder traversal algorithm traverses the parent node first,
then traverses the left and right tree by calling the itself on each.}

traversal, while \ak{^} and \ak{v}
pre-order traverse only non-inline non-list expressions, mirroring the visual
line motion a normal cursor might make.

\section{Discoverability}
\setlength\intextsep{0pt}
\begin{wrapfigure}[21]{R}{5.5cm}
\includegraphics[width=5.5cm]{figures/menu.png}
\caption{Kale's Context Menu}
\end{wrapfigure}
Most commands can be accessed in at least three ways, through their dedicated
keyboard shortcut, the context menu and the top-level editor menu. In case the
user is keen on using the keyboard shortcuts but forgot a specific command,
each editor menu-item shows the corresponding keyboard shortcut, and each
shortcut can be triggered whilst the context menu is open. These keyboard
shortcut indicators are also placed throughout Kale to help with discovering
shortcuts for the Clipboard List or the Function Search menu.


\section{High-level manipulation}

\subsection{List merging}

\subsection{Clipboard list}
\begin{wrapfigure}[11]{R}{0.pt}
	\includegraphics[width=5.5cm]{figures/clipboard}
	\caption{The Clipboard List}
\end{wrapfigure}

Relinquishing control over the elementary elements of a program might
potentially make more high-level manipulation problematic. To remedy this
Kale needed to provide a better way of transforming expressions. Andrew Blinn's
Fructure \cite{Fructure} environment tackles this by letting user colour
expressions, then
using a special transformation mode where the user can enter new expression or
use one of the previously coloured ones. While imbued with a certain sense of
mathematical purity, this approach deals poorly with more complex refactorings
and requires colour vision.

Kale's solution to this comes in the form of the \textbf{Clipboard List}. The
Clipboard List is a stack of expressions, shown on the right-hand side of the
screen. Kale provides a \hyperref[cmd:copy]{\say{Copy}} \keys{C} command, which
copies
the
currently
selected expression to the top of the stack. To facilitate more destructive
refactoring, Kale provides a palette of deletion commands:

\begin{itemize}[noitemsep]
	\item \hyperref[cmd:delete]{\say{Delete}} \keys{\backspace}
	\item \hyperref[cmd:cut]{\say{Cut}} \keys{X}
	\item \hyperref[cmd:delete_blank]{\say{Delete and Add Space}} \keys{R}
	\item \hyperref[cmd:cut_blank]{\say{Cut and Add Space}} \keys{S}
\end{itemize}

\subsection{Smart space}
%TODO: Show a sequence explaining how this works.
%TODO: Mention how this isn't a problem in scratch since you can't add new
% arguments.
In contemporary programming environments various punctuation marks are used to
create new expressions. This presents a challenge for an editor like Kale, how
to create new expressions. At first this was implemented as a set of
expression kind specific operations like \say{Create new child}, \say{Create
new sibling}, and \say{Create new line}. However this proved unintuitive as
unlike classical punctuation, the keys for these operations did not correspond
to any character on the screen, thus making it hard to memorise the shortcuts.

The solution Kale is using is named \textbf{Smart Space}. Smart space is a
high-level operation that attempts to perform a reasonable action no matter the
selection.
\begin{itemize}[noitemsep]
	\item If a function is selected, create a new child space in the first
argument.
	\item If a space is selected, use the \hyperref[cmd:up_down]{\say{Move Up}}
	\keys{\shift + P} operation.
	\item Otherwise create a new sibling space to the right of the selection, for
example creating a new argument.
\end{itemize}

Note that this does not cover \say{Create new line} operation, so this option
is still exists as
\hyperref[cmd:new_line]{\say{New Line Below}} \keys{N} /
\hyperref[cmd:new_line]{\say{New Line Below}} \keys{\shift + N}.

\begin{figure}
	\begin{minipage}{0.5\linewidth}
	\centering
	\includegraphics[width=\linewidth]{figures/clipboard.png}
	\caption{First.}
	\end{minipage}
	\qquad
	\begin{minipage}{0.5\linewidth}
	\centering
	\includegraphics[width=\linewidth]{figures/clipboard.png}
	\caption{First.}
	\end{minipage}
\end{figure}

\section{Novice Users}
A very powerful operation for manipulating programs is drag and drop. It allows
for direct manipulation of Kale expressions, without requiring any knowledge of
the keybindings.

\section{Professional users}

%TODO: Talk about the different expr types

High level operations and complex programs necessitate an intelligent layout
engine which is able to cleanly layout arbitrary code.
%TODO: Write a quick slurp program

The conditions for an inline expressions are as follows:

\begin{itemize}[noitemsep]
	\item Expression is a literal\footnote{In the future this might not be the
only condition as more literal types get added}
	\item Expression is a variable name
	\item Expression is a call with no arguments
	\item Expression is a call and the following are true
	\begin{enumerate}[noitemsep]
		\item Every argument is also inline
		\item The sum total length of the arguments is below 300 pixels
		\item The hight of the expression tree of every argument is below 4
	\end{enumerate}
\end{itemize}

\section{Font choice}

\chapter{Implementation}

\chapter{Evaluation}

Kale good, text editing bad.

\chapter{Conclusion}

\section{Feasibility of visual programming environments}

\section{Future Work}

\begin{itemize}
	\item 
\end{itemize}

\endgroup % Ends the clearpage re-definition group
\clearpage
\renewcommand*{\bibfont}{\raggedright} % Make the reference ragged right.
\bibliographystyle{unsrtnat} % List in order of mention.
\bibliography{bibliography.bib}

\begin{appendices}
	\chapter{Kale commands}
\newcommand{\shortcut}[3]{\section[#1]{#1 \hfill #2}\label{cmd:#3}}
\newcommand{\pskip}[1]{{\parskip=1em\par\noindent #1}}

Commands different by only a shift key \keys{\shift} are closely
related. Commands that change the selection are usually not listed in the
context menu. Where applicable mnemonics are represented by a \textbf{B}old
letter.

\shortcut{Up / Down}{\ak{^} / \ak{v}}{up_down}
Change the selection to the previous or next expression by pre-order traversal.
This means that by running a command repeatedly, you will move through every
possible selection to either side of the cursor. If there is no previous or
next expression, the current selection is preserved.
\pskip{Alternative shortcuts: \keys{K} / \keys{J} (based on the Vim editor).}

\shortcut{Left / Right}{\ak{<} / \ak{>}}{left_right}
Change the selection to previous or next non-inline non-list expression by
pre-order traversal. This approximates moving between expressions that visually
resemble lines, or broken up lines. If there is no previous or next expression,
the current selection is preserved.
\pskip{Alternative shortcuts: \keys{H} / \keys{L} (based on the Vim editor).}

\shortcut{Add Space or Move Space Up}{\keys{\SPACE}}{smart_space}
This is also known as the \say{smart space}. If the currently selected
expression is a blank, \hyperref[cmd:move_up]{\say{Move Up}} is ran on the
current selection. If the selection is a function call, a new child space
is added in the first argument. Otherwise a new sibling space is inserted to
the right of the current selection. 

Because this command will not create a new sibling to function calls, the 
\hyperref[cmd:new_line]{\say{New Line Below /Above}}
\keys{N} / \keys{\shift + N} 
command is also useful to know.

\shortcut{Edit}{\keys{\return}}{edit}
This can also be invoked by double clicking on an expression. Edits the text
inside the expression.

If the expression is a space, instead opens a new menu,
letting you select the type of expression with which to replace the space. 
Once a selection is made editing proceeds as normal, with the exception of
editing function expressions, which when the edit is complete, will create
an appropriate number of spaces for their arguments.

If the expression is a list, no action is performed.

\pskip{Alternative shortcut: \keys{E}}

\shortcut{Copy}{\keys{C}}{copy}
Copy the currently selected expression to the top of the Clipboard Stack

\shortcut{Select Parent}{\keys{P}}{parent}
Select the parent of the currently selected expression. If a parent does not
exist, the current selection is preserved.

\shortcut{Move Up}{\keys{\shift + P}}{move_up}
Move the currently selected expression to be the last sibling of its parent.
If a parent does not exist, no action is performed.
This is similar to the "Barf" operation in the Emacs Par Edit mode.


\shortcut{New Line Below / Above}{\keys{N} / \keys{\shift + N}}{new_line}
Inserts a new list expression around the current selection. Note that "list
merging" is performed, merging immediate list children of a list with their
parent. This means that, for example, invoking this command on an expression
which already has a list as its parent, will not create a new list, instead
appending a new sibling to the selection to the current list.

\shortcut{Delete}{\keys{\backspace}}{delete}
Delete the currently selected expression, potentially replacing it with a new
space if no other expression would remain in a function.

\pskip{Alternative shortcut: \keys{D}}

\shortcut{Cut}{\keys{X}}{cut}
\hyperref[cmd:copy]{Copy} the currently selected expression, then
\hyperref[cmd:delete]{delete} it. Similar to the \keys{\ctrl + X} command in
text editors.

\shortcut{Delete and Add Space}{\keys{R}}{delete_blank}

Delete the currently selected expression, replacing it with a new space. This
helps helps if you want to completetly \textbf{R}eplace an expression with
something new.

\shortcut{Cut and Add Space}{\keys{S}}{cut_blank}
\hyperref[cmd:copy]{Copy} the currently selected expression, then perform
\hyperref[cmd:delete_blank]{\say{Delete and Add Space}}. This helps if you want
to \textbf{S}huffle expressions around, replacing the current selection with
something new, but using the old expression somewhere else.

\shortcut{Open Definition}{\keys{O}}{open_def}
\shortcut{New Argument Before / After}{\keys{I} / \keys{\shift + I}}{new_arg}
\shortcut{Comment}{\keys{Q}}{open_def}
\shortcut{Disable}{\keys{\textbackslash}}{open_def}
%TODO: Add the space transformation commands.
\end{appendices}

\end{document}
