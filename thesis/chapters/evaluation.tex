\chapter{Evaluation}

With the currently implemented features, Kale forms a completely
usable visual programming environment. Users are able to create
arbitrary expressions, and manipulate them with relative ease, without
ever encountering a single syntax error. Kale is stable and the layout
algorithm is quite efficient.

\section{Editing}
Many features work in tandem to make the editing experience as complete as it
is now. Drag and drop is simple to use, supports touch-screen devices and
common idioms like pressing~\keys{\ctrl} to copy instead of move an expression.
Most features can be accessed in variety of ways, catering to an extensive
set of skill levels, the 


\section{Editor features}
\begin{wrapfigure}[11]{r}{8cm}
\includegraphics[width=8cm]{dark_ui}	
\caption{Kale Dark theme}
\end{wrapfigure}

Kale provides a substantial set of features one would expect from a modern
programming editor. Auto-complete suggestions are provided when typing
function expressions. Functions can be quickly opened or created by using an
easily accessible fuzzy matching search box. The editor is available in both
Light and Dark themes, to match the taste of the user and the \ac{ui} is
reasonably polished.

\section{User interface}
On touch-screen devices, buttons and menus follow
\fnurl{Apple Human Interface Guidelines}
{https://developer.apple.com/design/human-interface-guidelines/ios/visual-design/adaptivity-and-layout/}
44pt x 44pt tappable area recommendations. By using the HTML
\fnurl{Pointer Events}{https://www.w3.org/TR/pointerevents/} API,
all reasonable input devices are support for drag and drop, including dragging
multiple expression simultaneously should the user so desire.

\section{Workflow}
Modern editors often provide a tabbing system, to allow
users to work on several files at once. While Kale's editing experience
is completely function focused, a single function stack makes it hard
to develop larger projects. A tabbing system, switching between several
different function stacks would greatly help scaling Kale to work on bigger,
more challenging projects.

\section{Kale language}
The Kale language right now really serves as a placeholder for a more complete
programming language Kale might operate on.
It is sufficient for demonstrating the broader editing experience, but is somewhat
hindered by the fact that Kale does expose an interface to set function parameters
on user-defined functions. Developing the language more might further help to
demonstrate the unique features Kale can provide. Even better, the Kale language
could instead consist of a simple translation layer over an existing, established
language, helping to broaden the appeal of the new editing paradigm. 


\section{Implementation}
React proved to be excellent choice as the \ac{ui} library. While there exists a
certain impedance mismatch between the Kale layout algorithm and React's programming
style, in practice this wasn't much of an issue, given how custom Kale's layout
algorithm is. Typescript and React are mature technologies that made much of
the rapid iteration essential to developing Kale into the editor it is today possible.

%TODO: Cite this all.
An earlier version of Kale was written in the Rust programming language. Rust is a
modern, powerful and safe programming language, which recently gained a WebAssembly
compilation target. Unfortunately Rust's immature web-library ecosystem,
combined with its strict memory model made it hard to implement a complete editor in.
