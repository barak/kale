\clearpage
\chapter{Implementation}

\section{Layout engine}

\begin{figure}[t]
\includegraphics[width=\linewidth]{debug_overlay}
% No idea why I need to protect this
% https://lookherefirst.wordpress.com/2008/04/28/citations-within-a-caption-in
% -latex/
\caption{Kale's internal layout information\protect\footnotemark}
\label{fig:layout}
\end{figure}
\footnotetext{The layout debug overlay is always available within Kale,
to access it, right click any expression, then press \keys{\ctrl}
(or \keys{\Alt} on macOS) to reveal a new hidden
\say{Toggle the Debug Overlay} option}
\medskip

A key component of the layout engine is the notion of \say{inline} expressions.
An inline expression is one which can be easily displayed inside another inline
expression. In \autoref{fig:layout} inline expression are those with a
\textcolor{blue}{blue} outline, those with a \textcolor{red}{red} outline are
non-inline.

\hyperref[expr:literal]{Literals}, \hyperref[expr:blank]{blanks} and
\hyperref[expr:variable]{variables} are always inline, while
\hyperref[expr:list]{lists} are always non-inline.
For \hyperref[expr:function]{function calls} a heuristic-driven algorithm is
used to determine the inline status. The |isCallInline| algorithm, determines
a function call to be inline if:

\begin{enumerate}[noitemsep]
	\item No comment exists on the function call.
	\item Every argument is also inline.
	\item The sum total length of the argument widths is below 300 pixels.
	\item The hight of the expression tree of every argument is below four.
\end{enumerate}

Non-inline function calls are broken up, and their underline stack (explained
\hyperref[layout:underlines]{below}) is rendered onto the screen, while inline
calls continue to build up the underline stack.

\subsection{Data structure}

Keeping track of all the data required by the layout algorithm
including enabling in-line editing and supporting drag and drop requires a
complex layout data structure.
Because parent expressions like function calls need access to their children's
layout information to make layout decisions, it is not sufficient to simply
treat Kale expressions as React components to be rendered. Instead the
\texttt{Layout} class is responsible for keeping track of state required
of each expression and \ac{svg} elements by their parents.

\begin{Verbatim}[samepage]
class Layout {
    size: Size;
    nodes: ReactNode[] = [];
    underlines: Underline[] = [];
    areas: Area[] = [];
    inline = false;
    isUnderlined = false;
    expr: Expr | null = null;
    partOfExpr: Expr | null = null;
    text: Optional<TextProperties>;
}
\end{Verbatim}

\newcommand{\field}[1]{
	\paragraph{\texttt{#1}}
	\label{layout:#1}
}
\field{size} A key component of effectively laying out expressions is
keeping track of their size. Note that since \ac{svg} elements have no
inherent size, this is simply a suggestion of what the predicated size
of a layout element will be.

\field{nodes} The nodes array is responsible for aggregating all the
\ac{svg} nodes rendered to up to this point.

\setlength{\columnsep}{5ex}
\begin{wrapfigure}[7]{R}{30ex}
\vspace*{-\baselineskip}
\begin{Verbatim}[samepage]
interface Underline {
    level: number;
    offset: number;
    length: number;
}
\end{Verbatim}
\caption{The underline interface}
\end{wrapfigure}
\field{underlines} Kale's inverted stack of underlines means individual
expressions cannot know at what level and at what depth their underline will
be rendered. Instead the underlines to be drawn at the first non-inline
function call parent are lazily kept track of during the layout process. With
each parent expression laid
out, the level and the offset relative to the parent expression is updated.
Once underlines are finally drawn, the stack is cleared.

\field{areas} Expression areas are the aggregation of all the layout element
data collected up this a certain point in the rendering process. It has a
variety of uses throughout Kale including, drag and drop hit-testing and inline
data used for the~\ak{^} and~\ak{v} traversal. Kale supports two kinds of
areas: expression areas and \say{gap} areas used for implementing
\hyperref[impl:dnd]{drag and drop}.

\field{inline} Stores the inline status of an layout element as explained
above.

\field{isUnderlined} Not every inline layout element needs to create
a new underline. Kale
chooses to not underline atomic expressions like spaces, literals and variable
names.

\field{expr}

\field{partOfExpr} Drag and drop \hyperref[layout:areas]{gap areas} for most
expressions allow inserting a new dropped expression between sibling
expressions. However, when using drag and drop over function call expressions
it is desirable to be able to create new \emph{child} expressions. This field
complements the |expr| field, and allows the function calls's text to have a
drop areas where one normally would not exist.

\field{text} Creating the \hyperref[soln:field_editing]{field editor} requires
re-creating the same text style as used during layout, this field is set by the
text layout methods to keep track of the exact text-styling used.

\subsection{Text metrics}

Creating the \texttt{Layout} data structure for text elements requires
information on the metrics of piece of rendered text. Unfortunately the current
state of web text-metrics API leaves a lot to be desired. While
the |<canvas>| element provides a seemingly comprehensive
\fnurl{TextMetrics}
{https://developer.mozilla.org/en-US/docs/Web/API/TextMetrics} API.
In reality the vast majority of the metrics systems like Kale might
be interested in consuming are currently only available in the latest
browsers, behind experimental flags.


Instead Kale takes the same approach as the Scratch Blocks \cite{ScratchBlocks}
library, using \ac{svg}'s |getComputedTextLength| API and an invisible
|<svg>| element onto which new pieces of text are rendered. 



\section{Drag and drop}
\label{impl:dnd}
\begin{wrapfigure}[6]{r}{0pt}
\includegraphics[width=7cm]{dnd_gaps}
\caption{The expression from \autoref{fig:layout} showing drag and drop
insertion slots.}
\end{wrapfigure}
Diagram.




\section{Expression IDs}

Mention the effect on the selection across editors.




\section{Workspace}

Mention the undo stack.




\section{Deployment}
Any code committed to Kale's Github repository%
\footnote{\url{https://github.com/mgoszcz2/kale}}
is automatically built by the Travis \ac{ci} service, which runs tests, bundles
and optimises code using
\href{https://webpack.js.org/}{webpack}
and deploys the resulting site to
\href{https://pages.github.com/}{Github Pages}%
.
