\chapter{Introduction}

\section{Motivation}
Syntax errors are a fact of life in the programming industry. Programmers from
novice \citep{Denny2011} to professional spend significant time fighting or
avoiding syntax errors. It is clear a visual programming environment such
%TODO: Cite scratch, cite syntax errors among professionals
as Scratch could eliminate syntax errors. There exists a large body of existing
visual programming environments \citep{Beldie1983}, but most focus exclusively
on making programming accessible to children and young adults. Most
professional tools focus instead of error-detection, ignoring visual means of
editing programs as slow and cumbersome. The Kale system is an attempt to show that it is possible to create a
visual programming environment that is friendly to novices, but powerful enough to be
taken seriously by professionals. 

Kale consists of a web based visual programming environment, designed
from the ground up to fit many skill levels. It demonstrates that a
drag-and-drop/blocks style interface can coexist with a keyboard-driven
professional-focused editing experience. It generates usable code,
runnable from within the interface.

\section{Approach}

%TODO: Write these out
\begin{itemize}
	\item Rough out design
	\item Implement the rendering engine
	\item Keyboard shortcuts
	\item Drag and drop functionality
	\item Working interpreter; which language?
	\item Editor support for refactoring
\end{itemize}

%\section{Metrics}
\section{Project}

As part of the development process I contributed patches to two different
open-source projects, including adding \texttt{\#rrggbbaa} notation support
to the Popmotion library \citep{github-pr-popmotion} and updating
\say{styled-components} TypeScript typings to the a new major 5.0
version \citep{github-pr-styled-components}.
