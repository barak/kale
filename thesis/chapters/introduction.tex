\chapter{Introduction}

\section{Motivation}
Syntax errors are a fact of life in the programming industry. Programmers from
novice \citep{Denny2011} to professional spend significant time fighting or
avoiding syntax errors. It is clear a visual programming environment such
%TODO: Cite scratch, cite syntax errors among professionals
as Scratch could eliminate syntax errors. There exists a large body of existing
visual programming environments \citep{Beldie1983}, but most focus exclusively
on making programming accessible to children and young adults. Most
professional tools focus instead of error-detection, ignoring visual means of
editing programs as slow and cumbersome. The Kale system is an attempt to show that it is possible to create a
visual programming environment that is friendly to novices, but powerful enough to be
taken seriously by professionals. 

Kale consists of a web based visual programming environment, designed
from the ground up to fit many skill levels. It demonstrates that a
drag-and-drop/blocks style interface can coexist with a keyboard-driven
professional-focused editing experience. It generates usable code,
runnable from within the interface.

\section{Approach}

The project consists of three major phases.

\subsection{Layout engine development}
In this phase, the goal is to implement a functional layout engine. Capable
of displaying static expressions and handling sufficiently complex code.

\subsection{Establishing a workflow}
The next phase consists of adding editing capabilities to the project.
These will be very keyboard-driven, to easily test out new ways of editing
code and find any major flaws.

\subsection{Simplifying the workflow}
Once completed the final goal will be to simplify the editing experience by
presenting the different editing operations in easy-to-use and accessible ways,
such as mouse or touch driven editing.

\section{Project}

\begin{itemize}
	\item The Kale visual programming environment can edit arbitrary expressions,
	using both mouse-driven and keyboard-driven workflows.
	
	\item A simple interpreted programming language always testing the editing
	experience on functioning examples.
	
	\item As part of the development process I contributed patches to two different
	open-source projects, including adding \texttt{\#rrggbbaa} notation support
	to the Popmotion library \citep{github-pr-popmotion} and updating
	\say{styled-components} TypeScript typings to the a new major 5.0
	version \citep{github-pr-styled-components}.
\end{itemize}