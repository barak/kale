\chapter{Technical Background}

\section{Topic material}

\subsection{Block based languages}

The MIT Media lab's Scratch \citep{Maloney2010} is probably the most broadly
successful block-based programming languages.

\subsection{Frame based editing}

Another attempt at mitigating syntax errors has been made in the form of
\say{frame based editing}.  \citet{Kolling2017} created a new
editing environment based around the concept of frames, it serves a middle-ground
between Kale and text-based editing.

\section{Technical Material}

\subsection{Command Composition}
Composing user commands is an important concern for any code editor.
\citet{Chodarev2016} discussed this at length, pointing out that casual
text-based editing environments rely on simple commands that operate only on
elementary objects, manipulating programs a single character at a time. However this
approach is quite inefficient, leading applications to adopt a secondary set of
ad-hoc commands for performing specific combinations of operations and
high-level objects (like deleting a word). One proposed alternative is
Vim-like shortcuts, where the user combines a smaller set of motion and action
commands to work on high-level objects.

\subsection{Proportional fonts}
Studies have shown that
proportional (variable-pitch) fonts are
faster to read than their \texttt{monospaced} counterparts \citep{Campbell1981,
Beldie1983}. One notable example
of a proportional font being used typeset code is the C++ Programming Language
book by Bjarne Stroustrup.

\subsection{TypeScript}
TypeScript is a structurally and gradually typed superset of JavaScript being
developed at Microsoft \citep{Typescript}. It provides stronger type-safety
guarantees than JavaScript and allows programs to use modern and experimental
JavaScript features such as class properties or optional chaining. 


\subsection{React and related libraries}
To ease the burden of manually manipulating the \ac{dom}, Kale uses the React
JavaScript library \citep{React}. React is a self-proclaimed \say{JavaScript
library for making user interfaces} being developed by Facebook. It
allows the program to write rendering
code in a functional fashion, relying a \say{Virtual \ac{dom} to make \ac{dom}
updates efficient}. Together with \ac{jsx}, a JavaScript extension supported by
TypeScript, React makes it easy to write complex \acp{ui} for the browser.

\subsection{\acf{svg}}
\ac{svg} is a web technology which can be used to display vector images in the
browser. It is scriptable by JavaScript and styleable with CSS, making it the
perfect fit for dynamically rendering any highly custom \acp{ui}. It is used by
Scratch for rendering their blocks interface.

\subsection{Lisp}
Lisp is a family of programming languages, known for its regular syntax,
homoiconicity, and dynamic features.
