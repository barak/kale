\section{Future Work}

\begin{itemize}[noitemsep]
	\item Better drag and drop - inserting at the end of expressions
	\item Value of structured underlines?
	\item User studies and feedback
	\item Debugging capabilities
	\item Explore more the high-level operations afforded by treating functions as
	fundamental unit of organising files.
	%TODO: Right now there is no save at all, fix that.
	\item Ability to work on multiple projects
\end{itemize}

\subsection{Fluid entry}
\say{Fluid entry} is a concept for simplifying entering new expressions. Right
now to turn a space into another expression the user needs to either enter the
\hyperref[soln:space_popover]{\say{space popover}} or memorise a set of
shortcuts like
\hyperref[cmd:make_var]{\say{Make a Variable...}}~\keys{V} or
\hyperref[cmd:make_var]{\say{Turn into a Function Call...}}~\keys{F}.
These shortcuts are currently essential to efficently creating new Kale
expressions, 

\subsection{Removing modality}
Removing the \hyperref[soln:field_editing]{field editing} feature might be
worth investigation. This could be achieved by implicitly entering the field
editing mode when changing the selection. \ak{<}~and~\ak{>} could then always
move the cursor inside the current field, similarly to standard text editors,
once. If the cursor reaches the end of a field, the next expression would be
selected, following the current default behaviour.