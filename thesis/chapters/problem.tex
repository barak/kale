\chapter{The Problem}

The programming language which Kale users will be writing should map well to
the UI, but be powerful enough to be able to express complex programs. It need
not necessarily be an existing language, it fact it might not be desirable to
make it one.

A proficient user should be able to write programs and navigate Kale without
having to reach for the mouse. This might involve mapping Kale concepts to
standard shortcut keys as well as developing new ones. It should be possible
for users of existing programming environments to adjust to Kale with relative
ease. Programs written in Kale should be readable, balancing accessibility and
information density.

Novice users should be able to to construct Kale programs in a simple and
initiative manner, on every device form-factor, whether touch-screen or
mouse-driven. It should be simple for them to discover new functionality and
edit existing programs without knowledge of higher level operations.

Effectively manipulating Lisp programs often requires commands for high-level
hierarchical selections and actions, which are often difficult for users to
remember. Editors like Vim use the command composition
pattern to enable these actions, but this comes at a great cost to the learning
difficulty \cite{Chodarev2016}